\documentclass[11pt,a4paper]{article}
\usepackage[utf8]{inputenc}
\usepackage{amsmath}
\usepackage{amsfonts}
\usepackage{amssymb}
\title{Importing BibTeX citation databases into citation management tools}
\author{Jane Sandberg}
\date{\today}

\begin{document}
\maketitle

BibTeX is a file format for lists of bibliographic references.  It is generally used with \LaTeX documents, but it is also used in other tools because of its relative simplicity.  A list of references in a BibTeX file is known as a \emph{database.}

These instructions show you how to import such databases (found online, provided by a colleague, or generated by Google Scholar or other online tools) into two user-friendly citation management tools.

\section*{Importing citations into the Zotero Firefox plugin}
\begin{enumerate}
\item On the Zotero toolbar, click on the gear icon.
\item Select ``Import.''
\item Navigate to the .bib database you wish to import and select it.
\item The citations from the database will appear in a collection of their own.  You may then move these items to other collections or use them to generate bibliographies.
\end{enumerate}


\section*{Importing citations into Mendeley Standalone}
\begin{enumerate}
\item Click the ``add files'' icon in the top left of your screen.
\item Navigate to the .bib database you wish to import and select it.
\item The citations from the database will appear in the ``recently added'' section of your Mendeley library.
\end{enumerate}

\section*{Importing citations into RefWorks}
\begin{enumerate}
\item Click ``import'' option in the quick access menu on the left side of your screen.
\item Choose BibTeX as the Import Filter/Data Source.
\item Choose the database from which you downloaded the BibTeX database.  If it doesn't appear in the list, choose \LaTeX or ``multiple databases.''
\item Select the file that contains your BibTeX citations.
\item Click Import.
\end{enumerate}

\end{document}